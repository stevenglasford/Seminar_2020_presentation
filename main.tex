\documentclass{beamer}
\usecolortheme[RGB={0,102,51}]{structure}
\usetheme[height=9mm]{Rochester}
\usepackage{animate}

\usepackage{amsfonts,amssymb,amscd,amsmath,mathrsfs,amsthm}

\usepackage{tikz}
\usetikzlibrary{lindenmayersystems}
\pgfdeclarelindenmayersystem{A}{%
  \symbol{F}{\pgflsystemstep=0.6\pgflsystemstep\pgflsystemdrawforward}
  \rule{A->F[+A][-A]}
}

% Usetheme:
%
%\usetheme{Helsinki}

\author{Steven Glasford}
\title{Modeling a Fungal War on a Plant.}
\date{\today}




\begin{document}

% Frame 1
\begin{frame}
\maketitle
\end{frame}
%todo: add background photo from https://bygl.osu.edu/sites/default/files/field/image/tar%20spot%20punctatum%203%20%20S%20Fair%208-20-16.jpg to the title page it has a good photo of fungal pathagen

\AtBeginSection[]  % A frame titled `Content' is added before every new section starts.
{
\begin{frame}<beamer>
\frametitle{Content} % Title of the automatically added frame
\tableofcontents[currentsection]   % List all the structure components that call this frame (section, subsection, subsubsection,...; add `current')
\end{frame}
}


% Section `Introduction'; sections are defined outside of frames. They may lead to the automatic inclusion of additional frames, see above.
\section{Introduction}
\begin{frame}{Motivation and description}
    \begin{itemize}
        \item We want to see under which fitness circumstances a specific fungal pathogen. 
        \item By fitness circumstance we are referring to how the fungus acts in nature, which one is stronger and can spread the most on a plant.
        \item We want to see under which circumstance one defeats the other, which one wins, and the general conditions in which there is a tie.
    \end{itemize}
\end{frame}

\section{Math and Methods}
\begin{frame}{Methods}
\begin{itemize}
    \item Long story short, it is impossible to model this precisely, so we must use a numerical method to conduct the model.
    \item We used a program called ROC-HJ
\end{itemize}
\end{frame}

\begin{frame}{Mathematical statements}
    The following are some crucial functions that are needed to be passed into ROC-HJ
\end{frame}

\begin{frame}{Hamilton-Jacobi equations}
    The equations from the functions above create a system of Hamilton-Jacobi Equations
\end{frame}

\section{Execution and Code}
\begin{frame}{Code}
    This is how we converted the Hamilton Jacobi to be used in the computer
\end{frame}

\begin{frame}{Output}
    
\end{frame}

\section{Graphs and Analysis}
\begin{frame}{Graphs}
    
\end{frame}

\section{Conclusion}
\begin{frame}{Conclusion}
    
\end{frame}

\begin{frame}{Current Limitations}
    \begin{itemize}
        \item This is not super general
        \item One-seasonal
    \end{itemize}
\end{frame}

\end{document}