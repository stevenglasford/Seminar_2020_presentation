\documentclass{beamer}
\usecolortheme[RGB={0,102,51}]{structure}
\usetheme[height=9mm]{Rochester}
\usepackage{animate}

\usepackage{amsfonts,amssymb,amscd,amsmath,mathrsfs,amsthm}

\usepackage{tikz}
\usetikzlibrary{lindenmayersystems}
\pgfdeclarelindenmayersystem{A}{%
  \symbol{F}{\pgflsystemstep=0.6\pgflsystemstep\pgflsystemdrawforward}
  \rule{A->F[+A][-A]}
}

% Usetheme:
%
%\usetheme{Helsinki}

\author{Steven Glasford}
\title{Modeling a Fungal War on a Plant.}
\date{\today}




\begin{document}

% Frame 1
\begin{frame}
\maketitle
\end{frame}

\AtBeginSection[]  % A frame titled `Content' is added before every new section starts.
{
\begin{frame}<beamer>
\frametitle{Content} % Title of the automatically added frame
\tableofcontents[currentsection]   % List all the structure components that call this frame (section, subsection, subsubsection,...; add `current')
\end{frame}
}


% Section `Introduction'; sections are defined outside of frames. They may lead to the automatic inclusion of additional frames, see above.
\section{Introduction}

% \begin{frame}

% This is the first frame of the introduction.

% \end{frame}

% % Section `Results'
% \section{Definitions and Examples}

% \begin{frame}{Timed itemize command}  % Optional title argument

% This frame contains an example of a list that is internally broken down into 5 sub-frames.

% \begin{itemize}
% \item<2-> appears from slide 2 on
% \item<4-> appears from slide 4 on
% \item<3-> appears from slide 3 on
% \item<2-3> appears between slides 2 and 3
% \item<5-> appears from slide 5 on
% \end{itemize}
% \end{frame}

% \begin{frame}
% It is often sufficient to use the simpler syntax with the pause command before each item. This leads to the following structure.

% \begin{itemize}
% \pause \item Beamer is a wonderful class
% \pause \item One can $x$ make animations
% \pause \item One uses the \textbf{pause} command, for example
% \pause \item in order to bring in important ideas
% \end{itemize}

% \end{frame}

% \begin{frame}
% \begin{itemize}
% \item Thomas Stieltjes and his mass distributions
% \pause \item Used $\phi (x)$ to indicate total mass less than or equal to $x$.
% \pause \item Picture of this distribution on the real line
% \end{itemize}
% \end{frame}

% \begin{frame}

% A similar way of presenting slides can be achieved using the {\it uncover} and the {\it only} command. These are used outside of lists.

% \medskip

% \uncover<2->
% {appear from slide 2 on } \uncover<3-4> {appears from 3 to slide 4\\}
% \uncover<4>{appears on slide 4\\}
% \uncover<3->{appears from slide 3 on\\}
% \end{frame}

% \begin{frame}{Theorems}


% \begin{theorem} Sample theorem.
% \end{theorem}

% \end{frame}


% % Section `Examples'
% \section{Results}

% \begin{frame}

% \begin{block}{Block title}
% This is a block in blue
% \end{block}

% \begin{alertblock}{Alert-block title}
% This is a block in red
% \end{alertblock}

% \begin{exampleblock}{Example-block title}
% This is a block in green
% \end{exampleblock}
% \end{frame}

% \begin{frame}{Alerts in uncovering}

% \begin{itemize}
% \item<1-| alert@1> First point.
% \item<2-| alert@2> Second point.
% \item<3-| alert@3> Third point.
% \end{itemize}

% \end{frame}

% \begin{frame}

% \begin{align*}
% (z - w)^{-1}&\int_{-\infty}^{\infty} \big\{e^{2\pi wt} - e^{2\pi zt}\big\} G(t)\ dt \\
%  \uncover<2-| alert@2>{ &= 2\pi \int_{-\infty}^0 \bigg\{\int_t^0 e^{2\pi(z - w)u}\ du\bigg\} e^{2\pi wt} G(t)\ dt }\\
%   \uncover<3-| alert@3>{ &\qquad - 2\pi \int_0^{\infty} \bigg\{\int_0^t e^{2\pi(z - w)u}\ du\bigg\} e^{2\pi wt} G(t)\ dt} \\
%   \uncover<4-| alert@4>{  &= 2\pi \int_{-\infty}^0 \bigg\{\int_{-\infty}^u e^{2\pi wt} G(t)\ dt\bigg\} e^{2\pi(z - w)u}\ du }\\
%   \uncover<5-| alert@5>{ &\qquad - 2\pi \int_0^{\infty} \bigg\{\int_u^{\infty} e^{2\pi wt} G(t)\ dt\bigg\} e^{2\pi(z - w)u}\ du } \\
%   \uncover<6-| alert@6>{  &= 2\pi \int_{-\infty}^0 \bigg\{\int_{-\infty}^0 e^{2\pi w(t+u)} G(t+u)\ dt\bigg\} e^{2\pi(z - w)u}\ du } \\
%   \uncover<7-| alert@7>{  &\qquad - 2\pi \int_0^{\infty} \bigg\{\int_0^{\infty} e^{2\pi w(t+u)} G(t+u)\ dt\bigg\} e^{2\pi(z - w)u}\ du}
% \end{align*}

% \end{frame}

% \begin{frame}


% \begin{tabular}{lcccc}
% Class & A & B & C & D \\\hline
% X & 1 & 2 & 3 & 4 \pause\\
% Y & 3 & 4 & 5 & 6 \pause\\
% Z & 5 & 6 & 7 & 8
% \end{tabular}

% \end{frame}

% \begin{frame}[<+->]
% \begin{theorem}
% $A = B$.
% \end{theorem}
% \begin{proof}
% \begin{itemize}
% \item Clearly, $A = C$.
% \item As shown earlier, $C = B$.
% \item<3-> Thus $A = B$.
% \end{itemize}
% \end{proof}
% \end{frame}

% \begin{frame}{}
%     \begin{theorem}
%         $\Omega = \frac{\delta}{\Delta}$
%     \end{theorem}
%     \begin{animateinline}[controls,autoplay,loop]{2}
% \multiframe{8}{n=1+1}{
%   \begin{tikzpicture}[scale=10,rotate=90]
%     \draw (-.1,-.2) rectangle (.4,0.2);
%     \draw [blue,opacity=0.5,line width=0.1cm,line cap=round]
%       l-system [l-system={A,axiom=A,order=\n,angle=45,step=0.25cm}];
%   \end{tikzpicture}    
% }
% \end{animateinline}
% \end{frame}

\end{document}